
The system employs a 48-bit virtual address space with byte-addressable memory, using 4KB pages and 32-bit page table entries (PTEs) where 8 bits are reserved for protection and valid flags. The address translation uses a three-level page table hierarchy. Each of the three page table index fields is 10 bits. 

\begin{parts}
    \part [4] Given the virtual address $\mathtt{0x3456789ABC}$. Suppose the contents of the page table entries of the three-level page table are as follows:
    \begin{itemize}[leftmargin=*]
    \item The base address of the first-level page table is $\mathtt{0x1000}$.
    \item The base address of the second-level page table is $\mathtt{0x2000}$.
    \item The base address of the third-level page table is $\mathtt{0x3000}$, and the PTE with index 3 points to the Physical Page Frame (PPN) $\mathtt{0x4000}$. [4 points]
    \end{itemize}
    \vspace{0.5cm}
    Final physical address:\blank[3cm]{}.
    \vspace{0.5cm}
    
    \part[4] Calculate the total size of all second-level page tables. [4 points]
    \vspace{5cm} 
    
    \part[4] Consider an x86-64 system with 64-bit virtual addressing (48-bit usable), 4KB page tables, and 8-byte PTEs, supporting both 4KB pages (4-level: PML4$\to$PDPT$\to$PD$\to$PT) and 2MB huge pages (3-level, no PT). When using 2MB huge pages for application data, how is the virtual address partitioned? Specify bit allocations for PML4, PDPT, PD indices and page offset. [4 points]


\end{parts}
